%%%%%%%%%%%%%%%%%%%%%%%%%%%%%%%%%%%%%%%%%%%%%%%%%%%%%%%%%%%%%%%%%%%%%%%%%%%%%%%%
%2345678901234567890123456789012345678901234567890123456789012345678901234567890
%        1         2         3         4         5         6         7         8

\documentclass[letterpaper, 10 pt, conference]{ieeeconf}  % Comment this line out if you need a4paper

%\documentclass[a4paper, 10pt, conference]{ieeeconf}      % Use this line for a4 paper

\IEEEoverridecommandlockouts                              % This command is only needed if 
                                                          % you want to use the \thanks command

\overrideIEEEmargins                                      % Needed to meet printer requirements.

% See the \addtolength command later in the file to balance the column lengths
% on the last page of the document

% The following packages can be found on http:\\www.ctan.org
%\usepackage{graphics} % for pdf, bitmapped graphics files
%\usepackage{epsfig} % for postscript graphics files
%\usepackage{mathptmx} % assumes new font selection scheme installed
%\usepackage{times} % assumes new font selection scheme installed
%\usepackage{amsmath} % assumes amsmath package installed
%\usepackage{amssymb}  % assumes amsmath package installed

\PassOptionsToPackage{hyphens}{url}\usepackage{hyperref}
\usepackage{graphicx}

\title{\LARGE \bf
Investing in British American Tobacco South Africa 
}


\author{Thabiso Magwaza - 836403 \\ Kopano Malombo - 873087\\ Sifiso Mbhele - 912336 \\ Tshidiso Mosethle - 796826 }

\begin{document}



\maketitle
\thispagestyle{empty}
\pagestyle{empty}


%%%%%%%%%%%%%%%%%%%%%%%%%%%%%%%%%%%%%%%%%%%%%%%%%%%%%%%%%%%%%%%%%%%%%%%%%%%%%%%%
\begin{abstract}

\end{abstract}


%%%%%%%%%%%%%%%%%%%%%%%%%%%%%%%%%%%%%%%%%%%%%%%%%%%%%%%%%%%%%%%%%%%%%%%%%%%%%%%%
\section{INTRODUCTION}

The ability to invest in stocks is an advantageous one for students. This is because early investing allows the investor to reap the full benefits of compound interest. The investment can be allowed to mature for years in the case of students thus giving them a financially stable start to life after university \cite{earlyInvestment}. In the modern day, as technology has availed a wealth of information and digital platforms to make investing easier, investing has become less strenuous to learn. 

This report presents the details of a $R$100 000 investment into the British British American Tobacco company (BAT) company. This company is chosen because of it's stability over the past year despite allegations of bribery against it and also it's recent merger with Reynolds American Inc. which has made it the largest tobacco company in the global market. The growing market of e-cigarettes, which BAT has a strong and growing presents in, further confirms it a potentially profitable investment.

This paper first presents a background of the project followed by a discussion of the \textit{Consumer goods and services} sector of the Johannesburg Stock Exchange (JSE). A quarterly analysis of the BAT company is then presented followed by a description of it's future prospects that prove it to be an attractive investment. A short conclusion is then presented at the end of the report. 

\section{Background}

A group of four students has been tasked with investing $R$100 000 into a sector in the JSE. The money is to be invested in such a way that the capital growth is maximized over the next twelve months. An analysis of the sectors of the JSE is to be done so as to select the most stable sector to invest in thus maximizing the probability of capital growth. Analysis tools such as the JSE's All-share index (ALSI), economic outlook reports from various sources and companies' financial reports for the past year are used to select a stock that is predicted to maximize the capital growth of the investment over the next twelve months. The process used in selecting the stock involves selecting a well performing sector followed by selecting a well performing company in that sector then only investing in the company after confirming it's stability and positive economic outlook for the next twelve months. Appendix \ref{sec:flowChart} presents a flow chart of how the investment is selected. This project aims to give the students experience in using available tools to investigate the stability of a sector in the JSE, understand the effects of macroeconomic events on the value of stock, understand how to use a company's financial reports to determine it's stability and also to use a company's financial outlook to determine the likelihood of capital growth should an investment be made in it. 

\section{Investing in the British American Tobacco company in the JSE}

In order to make an investment, the sectors of the JSE are first analysed so as to assess their stability and growth. This analysis narrows the search for a stable share that has a high likelihood of capital growth as the share will most likely exist in the most stable sector that has grown consistently over the past year. The stability of a company is an important factor as the more stable companies are easier to analyse with a comfortable level of certainty. A more confident analysis gives the investor peace of mind on the truthfulness of a prediction. This section discusses the reasoning behind the choosing of the \textit{Consumer goods and services} sector and also the BAT company.


According to the 2017 budget review \cite{budgetReview} released by the Department of the National Treasury South Africa, the \textit{Consumer goods and services} sector was the main contributor to economic growth and job creation in 2016. This is found not to be a recent trend as an economic output presentation by STANLIB in 2013 \cite{stanlibReport} claims that although the South African economy had recovered after the 2008 crash, the recovery was mostly consumer based rather than production based. Consequently the \textit{Consumer goods and services} sector had outperformed all the other sectors and accounted for $70\%$ of the economy. This leads to the observation that the \textit{Consumer goods and services} sector and has grown steadily over the past year and is likely to continue to do so for the next year as the Department of the National Treasury South Africa predicts an economic growth of $1.3\%$ in 2017 and $2\%$ in 2018.

To further the investigation into the stability of the \textit{Consumer goods and services} sector the closing prices for the past year of four large companies in the sector are investigated. The stability of these companies further suggests the stability of the sector and presents four potential shares to invest in. Appendix \ref{sec:ClosingPrices} presents the closing prices for  the past year of BAT,  Capevin Holdings Ltd, Anheuser-Busch InBev SA NV and Tiger Brands Limited. Although some of the shares have decreased in share price over the past year the closing prices show a trend of stability from the four companies thus further confirming the stability of the sector. BAT tops the charts in the JSE top 40 \cite{top40} and it's current drop in share price provides an opportunity to potentially buy a profitable share for a cheap price. BAT is thus further investigated as a potential investment.
 
 
 \section{ Quarterly Analysis}	
 	The BAT company is the current leading company in JSE's top 40 Index. The major factor contributing to this observed economic success is the company's progressive organic performance relative to its counterparts in the \textit{Consumer goods and services} sector of the JSE.  With  an adjusted revenue, profit, margin and EPS growth of $14.7\%$, $15.8\%$ , $30 bps$ and $21.0\%$ respecitively over the first half of $2017$, BAT proves to be an attractive investment option in the long run\cite{interimResults}. Despite BAT's high organic perfomance, on August $16$, $2016$, BAT's share price experienced a $13\%$ drop because of disclosed claims of bribery by a BBC documentary which aired in $2015$ \cite{Jonathan_Weber}. Within a week after the share price decline,  BAT was able to recover $7\%$ of this loss\cite{interimResults}.  
 	It should be noted that over the past ten years, BAT has returned an average of $3.98\%$ per year dividend payouts.The current Payout Ratio(PR) for the stock is $74\%$, meaning that the dividends are sufficiently covered by BAT's earnings. The stock now yields roughy $3.62\%$ with a market cap of \pounds $138$ Billion\cite{BAT_partOfYourPort}. 
 	
\subsection*{Detailed quarterly analysis of BAT}

 	\begin{enumerate}
 		\item[1.]\textit{Quarter one: $01/07/2016$ - $ 30/09/2016$}
 		\\
 		The company announced a key board appointment. The appointee(Dr. Helmes) was said to have distinguished financial expertise judging from her extensive career  and international coorperate experience. She had held CFO positions in companies such as \textit{NXP} and  \textit{ProsiebienSat.1 Media} \textit{inter alia}\cite{Board_Appointment}. This board appointment boosted investor confidence in BAT's management team. During this quarter,  BAT received international recognitions including being ranked as number one in good governance and being the only tobacco company to be in the top 50 of the world's most diverse and inclusive companies by Thomson Reuters. International recognition attracts investors and raises the company's shareprice in the stock market. Appendix \ref{sec:QuarterlyAnalysis} shows how these milestones increased BAT's share price in the JSE despite the dominant downward trend of the share price. In terms of economic growth, BAT's:
 		\\
 		\hspace{2.00cm}i. Revenue grew strongly, by $10.2\%$ at current rate of exchange\cite{interimResults}.
 		\\
 		\hspace{1.8500cm} ii. Global brands performed exceptionally well, with cigarette volume up by $9.8\%\cite{interimResults}.$	
 		\\
 		This was a successful quarter for the BAT.
 		\item[2.]\textit{Quarter two: $01/10/2016$ - $31/12/2016$}
 		\\
 		BAT proposed a merge with Reynolds American Inc.\footnote{Reynolds American Inc. is in the top 5 best performing companies in the dowJones stock exchange.} (RAI) through the acquisition of the remaining $57.8 \% $ it did not already own. It valued RAI at $\$ 56.60$ per share of which $\$ 24.13$ would be in cash and $\$ 32.37$ would be in BAT shares \cite{Proposes_Merger}. This meant top RAI executives were offered seats in BAT's board of directors. The merging of these companies meant combined R $\&$ D capabilities to deliver a class pipeline of vapour and tobacco heating products across markets globally. This would then give BAT a competitive advantage since Next Generation Products(NGPs) would result and thus eventually making BAT to be the largest listed tobacco company by operating profit and net turn over. This proposal stimulated investments. Despite these encouraging news of physcal growth, BAT's share price in the JSE continued to drop(see Appendix \ref{sec:QuarterlyAnalysis}) until the beginning of December 2016 because of the aforementioned alleged bribery allegations  . From this month, BAT recovered the losses. This shows stability within the company because despite of bad publicity, the BAT's board remained intact and they launched an internal investigation regarding the bribery issue.
 		
 		\item[3.]\textit{Quarter three: $01/01/2017$ - $31/03/2017$}
 		\\
 		BAT and RAI reached an agreement concerning the aforementioned proposal. To the stakeholders of BAT, this merging is sure to create a trully global tobacco and NGP company which will yield a relatively high net turnover. According to BAT the acquisition will enable the company a balanced presence in high growth emerging markets combined with direct access to the attractive US market. 
 		
 		\item[4.]\textit{Quarter four: $01/04/2017$ - $30/06/2017$}
 		\\
 		BAT launched \textit{iGlo} in Canada, which is a product providing a cleaner smoking experience as opposed to conventional cigarettes, by heating instead of burning tobacco. This product has already gained a $6.7\%$ share of the tobacco market in Japan \cite{BAT_nextgen}. Meanwhile in RSA down-trading and the growth of illicit trading caused a dip on the volume and profits of the company, by $4.6\%$ and $14.0\%$ ($\pounds2.3bn$) respectively \cite{Board_Appointment} \cite{BusinessLives}.The market share of Dunhill and Benson \& Hedges continued growing although they became more than offset by Peter Stuyvesant, resulting in a down total market share\cite{Board_Appointment}.

 		

 	\end{enumerate}	

 	As of September $2017$,  BAT has acquired the $57.8\%$ it set out to acquire in the second quarter as previosly stated. This merging will bring BAT a higher inorganic performance in the long run. The production of NGP's and an access in the world's largest markets(i.e. the US and UK) gives BAT a competitive edge, this will see BAT becoming the largest tobacco company in the global market. BAT's current challenges which have an effect in their share price include the FDA's planned nicotin reduction in cigarettes and plain packaging from many governements in which the BAT has dealings with.


\section{Future prospects of British American Tobacco(BAT)}
BAT is responsible for the sale of $90\%$ of the cigarettes sold to around $6.5~million$ adult smokers in South Africa. BAT also possesses around 200 trademarks which has resulted in it being one of the biggest marketshares worldwide \cite{BAT_hist}. South Africa currently holds the third largest share of the profits generated by BAT worldwide with a percentage figure of $8\%$ \cite{BAT_hist}. Since the majority of their profits are generated internationally, BAT has reduced risk on political and economic events that take place in South Africa.

It has been reported that their volume of cigarretes drops by $1.2\%$ annually as a result of a decrease in the number of smokers \cite{BAT_nextgen}. This results in less cigarettes sold and a decrease in cost of production.The company growth also does not benefit from advertising due to the regulations in South Africa.

The company still experiences growth since the price of cigarretes increases annually by $4\%-6\%$ \cite{BAT_hist} which provides a potential sales growth of $3\%-4\%$. The acquisition of $57.8\%$ of Reynolds American Inc \cite{BAT_vap} also suggests that more growth can be expected from BAT in the next couple of years especially in the electronic cigarette market where they are planning on increasing their presence from $15$ to $40$ markets by the year 2020. The electronic cigarettes, which trade under the Vype trademark, include non-warming and non-burning cigarettes that provide less risky alternative to smoking.

The future venture of BAT into a sustainable pipeline of products while maintaing their high quality distribution shows a good deal of innovation and will play a very key role in their future growth. The BAT has also invested $\$1$ billion in the research and development of these next generation products which could possibly give them an edge over their closest competitor Phillips Morris due to the expected rise in e-cigarettes demand \cite{BAT_comp}\cite{BAT_nextgen}. The above evidence leads to a reasonable expectation that BAT will outperform markets in the future thus making it worthy of the $R100 ~ 000$ investment.

\section{Conclusion}

The \textit{Consumer goods and services} sector of the JSE is found to be a stable sector with constant growth over the past year as it was the main contributor to economic growth and job creation in 2016. The BAT company, being the largest company in the sector, is found to be stable. After the disclosed claims of bribery by a BBC documentary which caused a $13\%$ drop in it's share price, the company has been able to regain investor confidence and recover most of it's share price. BAT has managed to secure a merger with Reynolds American Inc. which makes it the largest tobacco company in the global market. This fact coupled with the growing market of e-cigarettes, which BAT has a strong and growing presence in, makes BAT and attractive share worthy of the $R100 ~ 000$ investment.


\bibliographystyle{plain}
\bibliography{bib} 

\cleardoublepage
\appendix

\subsection{Flow diagram of investment selection}


\begin{figure}[h!]
\centering
\includegraphics[scale=0.5]{decisions.pdf}
\caption{Flow diagram of investment selection}
\label{fig:flowChart}
\end{figure}

\label{sec:flowChart}

\subsection{Closing prices for the past year of shares from the Consumer goods and services sector}

\begin{figure}[h!]
\centering
\includegraphics[scale=0.43]{BATSA.JPG}
\caption{Closing prices for BAT \cite{BATSA}}
\label{fig:BASTA}
\end{figure}

\begin{figure}[h!]
\centering
\includegraphics[scale=0.43]{capevin.JPG}
\caption{Closing prices Capevin Holdings Ltd \cite{capevin}}
\label{fig:capevin}
\end{figure}

\begin{figure}[h!]
\centering
\includegraphics[scale=0.43]{inbev.JPG}
\caption{Closing prices for Anheuser-Busch InBev SA NV \cite{inbev}}
\label{fig:inbev}
\end{figure}

\begin{figure}[h!]
\centering
\includegraphics[scale=0.43]{TigerBrands.JPG}
\caption{Closing prices for Tiger Brands Limited \cite{tigerBrands}}
\label{fig:tigerBrands}
\end{figure}


\label{sec:ClosingPrices}

\cleardoublepage



\subsection{Quarterly Analysis}
\begin{figure}[h!]
	\centering
	\includegraphics[width=0.9\linewidth]{Capture}
	\caption{showing share price of BAT in the LSE (blue) and in the JSE (orange)}
	\label{fig:capture}
\end{figure}
\label{sec:QuarterlyAnalysis}

\end{document}

